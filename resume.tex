%% start of file `template.tex'.
%% Copyright 2006-2010 Xavier Danaux (xdanaux@gmail.com).
%
% This work may be distributed and/or modified under the
% conditions of the LaTeX Project Public License version 1.3c,
% available at http://www.latex-project.org/lppl/.


\documentclass[11pt,a4paper]{moderncv}
\usepackage{bibentry}
\usepackage[T1]{fontenc}
\usepackage{color}
%\usepackage[]{babel}
% moderncv themes
\moderncvtheme[green]{classic}
\usepackage[utf8]{inputenc}
\usepackage{wasysym}


% adjust the page margins
\usepackage[scale=0.92]{geometry}

\firstname{Noé}
\familyname{Gaumont}
\title{Ph.D. student at Université Pierre et Marie Curie} 
%\extrainfo{Permis B}

\address{13 rue Gustave Simonet}{ 94200 Ivry-sur-Seine} 
\mobile{+33 6.77.79.86.28}
\email{noe.gaumont@gmail.com}
%\photo[72pt]{"image"}


\nopagenumbers{} 

\begin{document}
\maketitle
\vspace{-1.2cm}

\section{Professional Experience}
\cventry{Since October 2013} {Ph.D. student} {Université Pierre et Marie Curie}{}{in the ComplexNetwork team, LIP6, Paris}
{
Community detection in link streams. Link stream provide a new way to understand temporal networks.
A link stream is a sequence of timed interactions between two entities, \emph{e.g.} email exchanges.
In this context, a community should be a dense sub-stream, \emph{e.g.} a discussion instead of a group of
friends.
}
\vspace*{0.2cm}
\cventry{Feb 2013 - July 2013}{End-of-studies internship}{Thales Air System in the Innovation Lab}{Rungis}{}{Study and optimization of flight plan predictions on specific way-points.\newline \textit{Key concepts:} machine learning, data extrapolation. \textit{Languages:} C++, R.}%

\cventry{Sept 2011 - Feb 2012}{Software developer internship}{Commissariat à l'énergie atomique (\textbf{CEA})}{Brétigny-sur-Orge}{}{Design and development of an algorithm able to generate quadrilateral mesh under a vector field constraint and geometric constraints.
\textit{Key concepts}: paving mesh generation, finite element. \textit{Language}: C++.}

%\cventry{Jan 2009 - Feb 2009}{Worker internship}{Sealed Air - quality department}{\'Epernon}{}{Product control within the scope of quality check and communication with clients.\newline}%

%\cvline{April 2010 - Aug 2010}{Exchange program in Germany at the Technische Universität Hamburg-Harburg (TUHH).\newline Introduction to the finite element method and to the structural aspect of planes.}

\section{Education}
\cventry{Sept 2008 - July 2013}{Engineering school}{Université de Technologie de Compiègne}{}{in Computer Science, Compiègne}
{IT project examples carried out during my university training:
%\begin{itemize}\renewcommand{\labelitemi}{$\; \; \; \; \;   \circ$}	
%	\item Development in C++ of meta-heuristics to solve the 2D bin packing problem under guillotine constraints.
%	\item Development of the simplex algorithm in scilab.
%	\item Decentralized chess game developed in Java with a group of \textit{23} people.
	%\item Development of a Gephi plugin to solve the multi-source vehicle routing problem with heuristics.
%	\item Conception of a Tower Defense game in C++ and Qt.\newline
%\end{itemize}
}
\cvline{June 2008}{\textbf{High school diploma in science}, specialty mathematics, with honors in lycée \textsl{Fulbert}, Chartres.}

\section{Technical skills}

\cvline{Mathematics}{Graph theory, complex systems, mathematical optimization, meta-heuristics, constrained programming, Markov chain, basics in cryptography.}

\cvline{Computer}{ \vspace*{-0.25cm}
\begin{description}
\item[Programming:]{C++, Python, Rust, Lisp,  PostgreSQL.}
\item[Software:]{Git/svn, Gephi/Tulip, Scilab.}
\item[Web:]{HTML, JavaScript, CSS, PHP.}
\end{description}\vspace*{-1cm}
}
\vspace*{-0.55cm}
%\tiny{\CIRCLE \LEFTcircle}
%\bibliographystyle{plain}
%\bibliography{library.bib}
%\bibentry{Gaumont2016}
%\bibentry{Gaumont2015}
%\bibentry{Gaumont2014}
\section{Publications}
\subsection{\hspace*{0.2cm}International workshop}
\cvline{[1]}{Noé Gaumont, Tiphaine Viard, Raphaél Fournier-S'niehotta, Qinna Wang and Matthieu Latapy. Analysis of the temporal and structural features of threads in a mailing-list. In \emph{Complex Networks VII}, Dijon, France. 2016. \emph{Acceptation rate: 23\%}}
\cvline{[2]}{Noé Gaumont, François Queyroi, Clémence Magnien and Matthieu Latapy. Expected Nodes: a quality function for the detection of link communities. In \emph{Complex Networks VI}, New-York, USA. 2015. Long version of [3]. \emph{Acceptation rate: 20\%}}
\subsection{\hspace*{0.2cm}National conference}
\cvline{[3]}{Noé Gaumont and François Queyroi. Partitionnement des liens d'un graphe : Critéres et Mesures. In \emph{Algotel - 16èmes Rencontres francophones sur les Aspects Algorithmiues des Télécommunication}, Ile de ré, France. 2014. \emph{Acceptation rate: 55\%}}
\cvline{[4]}{Noé Gaumont. Trouver des séquences de contacts pertinentes dans un flot de liens. In \emph{Algotel - 18èmes Rencontres francophones sur les Aspects Algorithmiues des Télécommunication}, Bayonne, France. 2016. Short version of [5]. \emph{Acceptation rate: 60\%}}
\subsection{{\Large Under review}}
\subsection{\hspace*{0.2cm}Journal}
\cvline{[5]}{Noé Gaumont, Clémence Magnien and Matthieu Latapy. Finding remarkably dense sequences of contacts in link streams. Submitted to \emph{Social Network Analysis and Mining}.}

%\bibliographystyle{apalike}
%\bibliography{my.bib}

%\section{teaching}
%\cvline{bla bla}{blabla}
\clearpage

\section{Talks}

%\cvline{}{Masterclass with Crowcroft january 2016.}
%\cvline{}{Rescom january 2016}
\cvline{}{"Bringing density to link streams reveals meaningful groups in contact traces" in workshop e-Young Researchers Network in Complex Systems. 2015: \url{http://cs-dc-15.org/e-tracks/global/\#yr}}

\cvline{}{"Tools to study link streams" in workshop  Outils d'analyse de la dynamique temporelle dans les réseaux in Toulouse, France. 2016}
\cvline{}{Tiphaine Viard and Noé Gaumont. LinkStreamViz: a drawing tool for link stream. In \emph{Workshop Dynamics On and Of networks}.}
\cvline{}{Several presentation alongside the CODDDE French research project.}
%\cvline{}{Presentation RNSC?}

\section{Scientific responsibilities}
\cvline{}{Member of the organizing committee of MARAMI 2014 and ASONAM 2015.}
\cvline{}{Reviewer for: SITIS 2015, WWW 2015, Algotel 2016, ICDE 2016 and Journal of Complex Networks.}
\section{Teaching and vulgarisation}
\cvitem{}{Basics of C (100h) and python (20h) programming}
\cvline{}{Introduction to database with PostgreSQL (20h)}

\cvline{}{Animations in various places to teach children and adults:
\newline $\circ$ Learning HTML and CSS with paper cubes and Thimble.
\newline $\circ$ What stay private when surfing on the internet?}

\cvline{}{Talk at the CoFestival, an inclusice event to convey technical and scientific knowledge. Title: "How research works in computer science" with Tiphaine Viard.}


\section{Training}
\cvline{}{School on structure and dynamics of complex networks (2 weeks)}
\cvline{}{Rescom 2014 : Network Science (1 week)}


\section{Language skills \hspace{4.75cm} Interest \hfill}
%\cvline{French}{Mother tongue}{}
\begin{minipage}{0.5\textwidth}
\cvline{English}{European level C1. Good working knowledge.\newline $\circ$ \textit{TOEIC score in 2012: 960/990}.}
\cvline{German}{European level B2. Basic knowledge.}
\end{minipage}
\hspace*{0.06\textwidth}
\begin{minipage}{0.4\textwidth}
\vspace*{-0.17cm}
Open-source software (Mozilla), sport (bouldering, badminton).
\end{minipage}

%\section{Interests}
%\cvline{}{\small Open-source software (Mozilla), strong involvement in association, sport(bouldering, badminton), video-game.}
 




\end{document}


%% end of file `template_en.tex'.