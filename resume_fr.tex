%% start of file `template.tex'.
%% Copyright 2006-2010 Xavier Danaux (xdanaux@gmail.com).
%
% This work may be distributed and/or modified under the
% conditions of the LaTeX Project Public License version 1.3c,
% available at http://www.latex-project.org/lppl/.


\documentclass[11pt,a4paper]{moderncv}
\usepackage{bibentry}
\usepackage{color}
%\usepackage[]{babel}
% moderncv themes
\usepackage{textcomp}
\moderncvtheme[green]{classic}
\usepackage[utf8]{inputenc}
\usepackage{wasysym}

\newcounter{publi}
\newcommand{\newpubli}[1]{\refstepcounter{publi}\label{#1}[\thepubli]}

% adjust the page margins
\usepackage[scale=0.92]{geometry}

\firstname{Noé}
\familyname{Gaumont}
\title{Chercheur en informatique spécialisé dans l'analyse de graphes} 
\extrainfo{Permis B}

\address{54 boulevard des minimes}{31200 Toulouse} 
\mobile{+33 6 77 79 86 28}
\email{noe@ngaumont.fr}
%\photo[72pt]{"image"}


\nopagenumbers{} 

\begin{document}
\maketitle
\vspace{-1.2cm}

\section{Expériences professionnelles}
\cventry{Oct. 2016 - Août 2018} {Postdoctorat} {Centre d’Analyse et de Mathématique Sociales (CAMS)}{}{résident à l'Institut des Sytèmes Complexes (ISC), Paris}
{
	Étude de la structure de réseaux tel que Twitter pour le projet \url{Politoscope.org}. Le but est de détecter les communautés de militants politiques sur Twiter mais surtout de suivre leurs évolutions au cours du temps et de caractériser leurs structures et leurs stratégies. Cela permet aux journalistes et aux citoyens de mieux comprendre l'organisation de Twitter sur les sujets politiques. \textit{Langages:} Rust, Scala/Spark, Python.
}
\vspace*{0.2cm}
\cventry{Oct. 2013 - Oct. 2016} {Doctorat} {Université Pierre et Marie Curie}{}{dans l'équipe ComplexNetworks, LIP6, Paris}
{
 Étude sur la détection de communautés dans les flots de liens. Les flots de liens sont un outil pour étudier les réseaux temporels.
 Un flot de liens est défini par une séquence d'interactions temporelles, les emails en sont un exemple. Dans ce contexte, une communauté est un sous-flot de lien définit par des liens et non par des n\oe uds. \textit{Langages:} C++, Rust, Python.
}
\vspace*{0.2cm}
\cventry{Fév. 2013 - Juillet 2013}{Projet de fin d'étude}{Thales Air System dans  l'Innovation Lab}{Rungis}{}{Étude et optimisation de la prédictibilité des points caractéristiques d’un vol\newline \textit{Concepts clés:} machine learning, data extrapolation. \textit{Langages:} C++, R.}%
\vspace*{0.2cm}
\cventry{Sept. 2011 - Fév. 2012}{Stage assitant-ingénieur}{Commissariat à l'énergie atomique (CEA)}{Brétigny-sur-Orge}{}{Conception et développement d'un algorithme générant un maillage quadrangulaire sous contraintes géométriques et d'un champ de direction.
\textit{Concepts clés}: paving mesh generation, finite elements. \textit{Langage}: C++.}

%\cventry{Jan 2009 - Feb 2009}{Worker internship}{Sealed Air - quality department}{\'Epernon}{}{Product control within the scope of quality check and communication with clients.\newline}%

%\cvline{April 2010 - Aug 2010}{Exchange program in Germany at the Technische Universität Hamburg-Harburg (TUHH).\newline Introduction to the finite element method and to the structural aspect of planes.}

\section{Éducation}
\cventry{Juillet 2013}{Diplôme d'ingénieur}{Université de Technologie de Compiègne}{}{en Informatique, Compiègne}
{%IT project examples carried out during my university training:
%\begin{itemize}\renewcommand{\labelitemi}{$\; \; \; \; \;   \circ$}	
%	\item Development in C++ of meta-heuristics to solve the 2D bin packing problem under guillotine constraints.
%	\item Development of the simplex algorithm in scilab.
%	\item Decentralized chess game developed in Java with a group of \textit{23} people.
	%\item Development of a Gephi plugin to solve the multi-source vehicle routing problem with heuristics.
%	\item Conception of a Tower Defense game in C++ and Qt.\newline
%\end{itemize}
}
\cvline{Juin 2008}{\textbf{Baccalauréat S-SVT}, spécialité mathématique, mention très bien au lycée \textsl{Fulbert}, Chartres.}

\section{Compétences}

\cvline{Mathématique}{Graph theory, complex systems, mathematical optimization, meta-heuristics, constrained programming, basics in cryptography.}

\cvline{Codage}{ \vspace*{-0.25cm}
\begin{description}
\item[Langages:]{Rust, C++, Python, Scala, Spark,  PostgreSQL.}
\item[Web:]{JavaScript/TypeScript, React, HTML, CSS.}
\item[Otuils:]{Git/Mercurial, Docker/Docker-compose, GitLab-CI Gephi/Tulip, Scilab.}
\end{description}\vspace*{-1cm}
}
\vspace*{-0.55cm}
%\tiny{\CIRCLE \LEFTcircle}
%\bibliographystyle{plain}
%\bibliography{library.bib}
%\bibentry{Gaumont2016}
%\bibentry{Gaumont2015}
%\bibentry{Gaumont2014}
\section{Langues \hspace{6.4cm} Intérêts personnels \hfill}
%\cvline{French}{Mother tongue}{}
\begin{minipage}{0.5\textwidth}
	\cvline{Anglais}{Niveau européen C1.\newline $\circ$ \textit{Score au TOEIC en 2012 : 960/990}.}
	\cvline{Allemand}{Niveau européen B2. Connaissances basiques.}
\end{minipage}
\hspace*{0.06\textwidth}
\begin{minipage}{0.4\textwidth}
	\vspace*{-0.17cm}
	Open-source software (Mozilla), vie privé sur internet, sport (escalade, badminton).
\end{minipage}

\section{Publications}


\subsection{\hspace*{0.2cm}Journal international}

\cvline{\newpubli{poli}}{ Noé Gaumont, Maziyar Panahi and David Chavalarias. Methods for the reconstruction of the socio-semantic dynamics of political activist Twitter networks: Application to the 2017 French Presidential elections.\url{https://hal.archives-ouvertes.fr/hal-01575456v3}}

\cvline{\newpubli{dense}}{Noé Gaumont, Clémence Magnien and Matthieu Latapy. Finding remarkably dense sequences of contacts in link streams. \emph{Social Network Analysis and Mining}, 6(1), 87. \url{https://hal.archives-ouvertes.fr/hal-01390043}}


\subsection{\hspace*{0.2cm}Journal national}
\cvline{\newpubli{hostilite}}{David Chavalarias, Noé Gaumont et Maziyar Panahi, (2019). Hostilité et prosélytisme des communautés politiques. Reseaux, n\textdegree~214-215(2), 67–107  \url{https://www.cairn.info/revue-reseaux-2019-2-page-67.html}}


\subsection{\hspace*{0.2cm}Conférence internationale}
\cvline{\newpubli{mail}}{Noé Gaumont, Tiphaine Viard, Raphaél Fournier-S'niehotta, Qinna Wang and Matthieu Latapy. Analysis of the temporal and structural features of threads in a mailing-list. In \emph{Complex Networks VII}, Dijon, France. 2016. \emph{Acceptation rate: 23\%}. \url{https://hal.archives-ouvertes.fr/hal-01345821}}
\cvline{\newpubli{expected}}{Noé Gaumont, François Queyroi, Clémence Magnien and Matthieu Latapy. Expected Nodes: a quality function for the detection of link communities. In \emph{Complex Networks VI}, New-York, USA. 2015. Long version of [\ref{expectedshort}]. \emph{Acceptation rate: 20\%}.\url{http://hal.upmc.fr/hal-01196796}}

\subsection{\hspace*{0.2cm}Conférence nationale}
\cvline{\newpubli{expectedshort}}{Noé Gaumont and François Queyroi. Partitionnement des liens d'un graphe : Critéres et Mesures. In \emph{Algotel - 16èmes Rencontres francophones sur les Aspects Algorithmiques des Télécommunications}, Ile de ré, France. 2014. \emph{Acceptation rate: 55\%} \url{https://hal.archives-ouvertes.fr/hal-00986216}}
\cvline{\newpubli{denseshort}}{Noé Gaumont. Trouver des séquences de contacts pertinentes dans un flot de liens. In \emph{Algotel - 18èmes Rencontres francophones sur les Aspects Algorithmiques des Télécommunications}, Bayonne, France. 2016. Short version of [\ref{dense}]. \emph{Acceptation rate: 60\%}. \url{https://hal.archives-ouvertes.fr/hal-01305118}}

\subsection{\hspace*{0.2cm}Autre}

\cvline{\newpubli{scimap}}{David Chavalarias, Maziyar Panahi and Noé Gaumont (2019). Politoscope. In K. Börner \& E. Record (Eds.), 15th Iteration (2019): Macroscopes for Tracking the Flow of Resources. \url{http://scimaps.org}}

\section{Présentations}

\subsection{\hspace*{0.2cm}Audience international}
%\cvline{}{Masterclass with Crowcroft january 2016.}
%\cvline{}{Rescom january 2016}

\cvline{\newpubli{mocda}}{Maximilien Danisch, Noé Gaumont and Jean-Loup Guillaume. \emph{A Modular Overlapping Community Detection Algorithm: Investigating the ``From Local to Global'' Approach} in Cologne Twente Workshop (CTW) . 2018: \url{https://papers-gamma.link/paper/33/A\%20Modular\%20Overlapping\%20Community\%20Detection\%20Algorithm:\%20Investigating\%20the\%20\%E2\%80\%9CFrom\%20Local\%20to\%20Global\%E2\%80\%9D\%20Approach}}

\cvline{\newpubli{poli-ccs}}{Noé Gaumont, Maziyar Panahi and David Chavalarias. \emph{Evolution of communities on twitter during the 2017 French presidential election} in Conference Complex Systems (CCS) . 2017. \url{https://easychair.org/smart-program/CCS'17/2017-09-18.html\#talk:47444}}

\cvline{\newpubli{poli-iscc}}{Noé Gaumont, Maziyar Panahi and David Chavalarias. \emph{Étude de la campagne Twitter “Ali Juppé”} in Colloque international sur L’élection présidentielle de 2017 et ses primaires : enjeux de communication politique. \url{http://www.iscc.cnrs.fr/spip.php?article2282}}

\cvline{\newpubli{link-viz}}{Tiphaine Viard and Noé Gaumont.\emph{LinkStreamViz: a drawing tool for link stream}. In \emph{Workshop Dynamics On and Of networks}. 2016. \url{https://project.inria.fr/netspringlyon/3-workshops-on-network-sciences/workshop-on-processes-on-and-of-networks/}}


\cvline{\newpubli{link-density}}{Noé Gaumont, Clémence Magnien and Matthieu Latapy. \emph{Bringing density to link streams reveals meaningful groups in contact traces} in workshop e-Young Researchers Network in Complex Systems. 2015. \url{https://cs-dc-15.org/e-tracks/global/\#yr}}



\subsection{\hspace*{0.2cm}Audience nationale}

\cvline{\newpubli{link-usecase}}{Noé Gaumont.\emph{Utilisation de flots de liens pour étudier les interactions temporelles}, 24e journées thématique de Rochebrune 2017}
\cvline{\newpubli{link-utils}}{Noé Gaumont. \emph{Tools to study link streams}, in workshop  Outils d'analyse de la dynamique temporelle dans les réseaux in Toulouse, France. 2016. \url{https://xsys.fr/wp-content/uploads/2016/09/journe\%CC\%81e-du-14-decembre.pdf}}

%\cvline{}{Presentation RNSC?}

\section{Responsabilités scientifiques}
\cvline{}{Membre du commité d'organisation de MARAMI 2014 et ASONAM 2015.}
\cvline{}{Relecteur pour: SITIS 2015, WWW 2015, Algotel 2016, ICDE 2016 et Journal of Complex Networks.}

\section{Projets scientifiques}
\cvline{2016-2019} {Participant au projet \emph{recommandation Algorithmique et Diversité des informations du web} (AlgoDiv) numéro ANR-15-CE38-0001 coordonné par Camille Roth (Sciences Po medialab).\newline
Comment mesurer, vérifier et contrôler la qualité des choix réalisés par les algorithmes du web lorsqu’ils décident de mettre en visibilité telle ou telle information ? Ce projet interdisciplinaire voudrait apporter une contribution scientifique  en procédant à une exploration empirique des effets que le guidage algorithmique exerce sur la diversité informationnelle du web, en construisant des outils destinés à mesurer cet impact et en proposant des alternatives aux utilisateurs.
\url{http://algodiv.huma-num.fr/}}
\cvline{2014-2017} {Participant au projet COmmunity Dynamics, Diffusion and Detection of Events (CODDDE) numéro ANR-13-CORD-0017-01 coordonné par Jean-Loup Guillaume (LIP6).\newline
Ce projet a pour but d'améliorer la compréhension de l'évolution des réseaux complexes réels sur trois thèmes: évolution des structures communautaires, diffusion d'informations et détection des changements de structure du réseau. \url{http://jlguillaume.free.fr/coddde/}}
\cvline{2014-2017}{Participant au projet REcursive QUEry and Scalable Technologies REQUEST  numéro O18062-444 financé par le 3e appel FSN \emph{Informatique en nuage – Cloud computing} coordonné par Catherine Gouttas (Thales).\newline
Ce projet a pour but  le développement d’algorithmes avancés, adaptés au traitement des données, massives, complexes, hétérogènes et dynamiques dans le cadre du \emph{Cloud}. \url{http://projet-request.org/}}

\section{Enseignements et vulgarisations}
\cvitem{L1 \& L2}{programmation impérative, éléments d'algorithme et structures de données en C (40h TD + 80h TP)}
\cvline{L1}{Éléments de programmation (Python) (20h TP)}
%\cvline{L2}{Programmation et structures de données en C (20h TP)}
\cvline{L2}{Introduction aux bases de données relationnelles (20h TP)}
\vspace*{-0.1cm}
\cvline{}{Animations volontaires:\newline \hspace*{0.1cm}
	\begin{minipage}{0.85\textwidth}\vspace*{0.2cm}
	\begin{itemize}
	\item[$\circ$] Participation à la journée \emph{Animation et médiation scientifique face aux infox}, \url{https://www.afastronomie.fr/face-aux-infox}.
	\item[$\circ$] \emph{Usage éthique et citoyen d'internet}, pour des membres du CIVAM.
	\item[$\circ$] \emph{Gestion de la vie privée sur le web}, pour enfants et adultes dans les bibliothèques.
	\item[$\circ$] \emph{Welcome to the light side of Big Data}, Pint of Science, Paris.\newline \url{https://ngaumont.fr/asset/Pint_Science/}.
	\item[$\circ$] \emph{Apprentissage de HTML et CSS avec des cubes en papier et Thimble}, pour  enfants et adultes dans les bibliothèques
	\item[$\circ$] \emph{Gestion de la vie privée sur le web}, pour enfants et adultes dans les bibliothèques.
	\item[$\circ$] Présentation du politoscope  lors de l'inauguration de l'exposition Tera Data à la Cité des Sciences. \url{http://www.cite-sciences.fr/fr/au-programme/expos-temporaires/terra-data/}
	\item[$\circ$] Présentation de projets de l'ISC dont le politoscope à Innovatives SHS, un salon de valoriation des sciences humaines et sociales à Marseille \url{http://innovatives.cnrs.fr/innovatives-shs-2017/exposition/article/expertise}
	\item[$\circ$] Tiphaine Viard and Noé Gaumont. \emph{De la découverte au partage dans la recherche en informatique}, au CoFestival, un événement inclusif les savoir-faire scientifiques et technologiques. \url{https://web.archive.org/web/20160111020002/http://cofestival.org/\#programme}
\end{itemize}\end{minipage}}
%\vspace*{-0.6cm}
%\section{Training}
%\cvline{}{School on structure and dynamics of complex networks (2 weeks)}
%\cvline{}{Rescom 2014 : Network Science (1 week)}






\end{document}


%% end of file `template_en.tex'.
